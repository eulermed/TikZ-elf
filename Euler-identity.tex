\documentclass[tikz,border=30mm]{standalone}
% !TeX program = luatex

%===This is the preambule I call in every file===

\usepackage{tikz}
\usepackage{xcolor}
\usepackage{pgfplots}
\usepackage{circuitikz}
\usepackage{tikz-3dplot}
\pgfplotsset{compat=newest}
\usetikzlibrary{arrows.meta, shapes.geometric, positioning, perspective, patterns.meta, decorations.pathreplacing, decorations.pathmorphing, decorations.markings, patterns, arrows.meta, shapes, shapes.geometric, decorations.text, angles, quotes,calc, 3d, math, circuits.ee.IEC,hobby, knots, intersections, through}


%=== The Euler Med Logo ===
%=== i.e. My signature ===

\usepackage{amsmath, amsfonts}
\makeatletter
\newcommand*\eulermed{{
\scalebox{3.3}{$\mathbb{E}$}\kern-1pt \scalebox{1.5}{u$\ell\varepsilon\rho$}\kern-55pt
\raisebox{19pt}{\scalebox{1.5}{$\mathcal{M}\varepsilon\delta$}}}
\@}
\makeatother

\usepackage{amsmath, amsfonts}
%defining new colors : 

\definecolor{eblue}{HTML}{00FFF6}
\definecolor{eblack}{HTML}{212529}
\definecolor{emauve}{HTML}{4C0033}
\definecolor{eviolet}{HTML}{E80F88}

%The code below is taken from this post tex.stackexchange.com/questions/59406/tikz-how-to-draw-an-isometric-drawing-in-tikz 


\makeatletter
\tikzoption{canvas is plane}[]{\@setOxy#1}
\def\@setOxy O(#1,#2,#3)x(#4,#5,#6)y(#7,#8,#9)%
  {\def\tikz@plane@origin{\pgfpointxyz{#1}{#2}{#3}}%
   \def\tikz@plane@x{\pgfpointxyz{#4}{#5}{#6}}%
   \def\tikz@plane@y{\pgfpointxyz{#7}{#8}{#9}}%
   \tikz@canvas@is@plane
  }
\makeatother  


\begin{document}



\pagecolor{emauve!60!black}
\color{white}

\tdplotsetmaincoords{70}{120}
\begin{tikzpicture}[>=stealth,tdplot_main_coords, scale=1.5]


\def\ang{3*pi}
\def\r{2*pi}
\def\abs{6*pi}

%The cosine plot 'Watch out there's a minus sign'
\begin{scope}[canvas is xy plane at z=0,]
\draw[dotted, line width=0.15mm, eblue,step=pi] (0,0) grid (\abs,\abs);
\draw[eviolet, line width=1.5mm, samples=200, variable=\t,domain=0:\abs] plot({-\r*cos(\t r)+\ang},\t);
\end{scope}

%The sine plot
\begin{scope}[canvas is yz plane at x=0,]
\draw[dotted, line width=0.15mm, eblue, step=pi] (0,0) grid (\abs,\abs);
\draw[eviolet, line width=1.5mm, samples=200, variable=\t,domain=0:\abs] plot(\t,{\r*sin(\t r)+\ang});

\foreach \i/\j in {pi/$\pi$ , 2*pi/$2\pi$, 3*pi/$3\pi$, 4*pi/$4\pi$, 5*pi/$5\pi$, 6*pi/$6\pi$}{
  \node[above, scale=5, eviolet] at (\i, \abs) {\j};
}

\end{scope}

%The circle 
\begin{scope}[canvas is zx plane at y=0,]
\draw[dotted, line width=0.15mm, eblue, step=pi] (0,0) grid (\abs,\abs);
\draw[eviolet, line width=1.5mm] (\ang ,\ang)circle(\r);
\end{scope}

%The parametric curve
\draw[dashed, eblue, line width=1mm,  variable=\t,domain=0:6*pi,samples=500] plot ({-\r*cos(\t r)+\ang},{\t},{\r*sin(\t r)+\ang});

\node[eblue, scale=5.5] at (4*pi,0,7*pi) {$e^{ix}=\cos x +i\sin x$}; 
\end{tikzpicture}
\end{document}
